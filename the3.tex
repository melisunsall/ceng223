\documentclass[12pt]{article}
\usepackage[utf8]{inputenc}
\usepackage{float}
\usepackage{amsmath}


\usepackage[hmargin=3cm,vmargin=6.0cm]{geometry}
%\topmargin=0cm
\topmargin=-2cm
\addtolength{\textheight}{6.5cm}
\addtolength{\textwidth}{2.0cm}
%\setlength{\leftmargin}{-5cm}
\setlength{\oddsidemargin}{0.0cm}
\setlength{\evensidemargin}{0.0cm}

%misc libraries goes here



\begin{document}

\section*{Student Information } 
%Write your full name and id number between the colon and newline
%Put one empty space character after colon and before newline
Full Name : Melis Ece ÜNSAL	 \\
Id Number :  2237865\\

% Write your answers below the section tags
\section*{Answer 1}
According to Fermat's Little Theorem:\\
We begin by considering the first p - 1 positive multiples of a; that is, the integers a, 2a, 3a, ..., (p - 1)a. None of these numbers are congruent modulo p to any other, nor is any congruent to zero. Indeed if it happened that \\
$r * a \equiv s * a (mod p),  1 \leq r < s \leq p - 1$
then, r $\equiv$ s (mod p), which is impossible because r and s are both between 1 and p - 1. Hence, the previous set of integers must be congruent modulo p to 1, 2, ... , p - 1. Multiplying these  together gives us\\
$a * 2a * 3a * ... * (p - 1) * a \equiv 1 * 2 * 3 * ... * (p - 1)(mod p)$\\
meaning\\
$a^{p-1} * (p - 1)! \equiv (p - 1)!(mod p).$\\
Cancelling (p - 1)! from both sides we obtain\\
$a^{p-1} \equiv 1 (mod p)$,(end of the proof of theorem).\\
When we put x instead of a, we obtain that y=(p-1) and also y$|$(p-1).

\section*{Answer 2}
For this statement to be true, right-hand side expression must be equal one of the divisors of 169.\\
Divisors of 169 are 13,1 and 169.\\
Now, we should check equality:\\
$\textbf{1.} 2n^{2}+10n-7 = 13 \rightarrow 2n^{2}+10n-20=0 \rightarrow n^{2}+5n-10=0 $\\
$\textbf{2.} 2n^{2}+10n-7 = 1 \rightarrow 2n^{2}+10n-8=0 \rightarrow n^{2}+5n-4=0 $ \\
$\textbf{3.} 2n^{2}+10n-7 = 169 \rightarrow 2n^{2}+10n-176=0 \rightarrow n^{2}+5n-88=0 $ \\
None of this equations has a root that is positive integer, so that we can say that any positive integer for n for the equation cannot divide 169.
\section*{Answer 3}
Let's say a $\equiv$ b(mod m*n), then $a=k*m*n+b$ for some k.\\
When we write $a-b=k*m*n$, then we can obtain that $a-b \equiv 0mod(m)$ since m is the divisor of the (a-b).\\
Also, we can obtain  $a-b \equiv 0mod(n)$ with the same idea.\\
All in all we prove that  $a \equiv bmod(m)$ and $a \equiv bmod(n)$. 


\section*{Answer 4}
Let's say j.(j+1).(j+2)....(j+k-1) is the function that is P(j).\\ 
\underline{Base Step}:\\
Take n=1 $\rightarrow \ \sum_{j=1}^{1} P(j) = 1.2.3.4...k =(?)$\\
$\dfrac{n.(n+1)..(n+k)}{k+1} = \dfrac{1.2..(1+k)}{k+1} $ (Cancel out (k+1)'s)\\
=  1.2.3.4...k    They're equal \ $\surd$\\
\underline{Inductive Step}:\\ \\
Let's say $\sum_{j=1}^{p} P(j) = \dfrac{p.(p+1)..(p+k)}{k+1} \ is \ true. $
We need to prove this for p+1:\\
$\sum_{j=1}^{p+1} P(j) = \sum_{j=1}^{p} P(j) + (p+1).(p+2).(p+3)....(p+k)$\\ \\
 $= \dfrac{p.(p+1)..(p+k)}{k+1} + (p+1).(p+2).(p+3)....(p+k) $ From the assumption above.\\
 When we do the rational addition, we obtain $\dfrac{(p+k+1).(p+1)..(p+k)}{k+1}$ and this equals:\\
 $\dfrac{(p+1).(p+2)..(p+k).(p+k+1)}{k+1}$ $\surd$ \\ End of the mathematical induction proof.
 


\section*{Answer 5}
\underline{Base Step}:\\
n=3 :  $H_{3} = 5H_{2} + 5H_{1} + 63H_{0}$\\
$= 5.5 + 5.3 + 63.1 = 103 < 7^{3}  \surd$  \\ \\
\underline{Inductive Step}:\\ 
Let's say $H_{3},H_{4},H_{5}.....,H_{k}$ is true, then we need to prove $H_{k+1}$ is true :\\ \\
$H_{k+1}=5H_{k}+5H_{k-1}+63H_{k-2}$\\
$\hspace*{1.5cm} \downarrow \hspace*{1.5cm} \downarrow \hspace*{1.5cm} \downarrow $\\
$\hspace*{1cm} \leq 5.7^k \hspace*{0.6cm} \leq 5.7^{k-1} \hspace*{0.4cm} \leq 63.7^{k-2} $  (From the assumption above) \\
$\hspace*{1.5cm} \downarrow \hspace*{1.5cm} \downarrow \hspace*{1.5cm} \downarrow $\\
$\hspace*{0.8cm} \leq 35.7^{k-1} +\hspace*{0.3cm} \leq 5.7^{k-1} +\hspace*{0.3cm} \leq 9.7^{k-1}  $\\
So,\\
$H_{k+1} \leq 49.7^{k-1}$\\
$H_{k+1} \leq 7^{k+1} \surd$


\end{document}